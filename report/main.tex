%!TeX program = xelatex
\documentclass{article}
\usepackage{papers}

%%-------------------------------正文开始---------------------------%%
\begin{document}

%%-----------------------封面--------------------%%
\maketitle

%%------------------摘要-------------%%
\begin{abstract}
本次测试报告旨在验证 Chromium 在 RISC-V 平台上的可用性,进行了 Chromium 在 Milk-V Pioneer Box 和 Sipeed LicheePi 4A 两个 RISC-V 典型平台上的测试。测试涵盖了多个 Linux 发行版和 Chromium 版本,通过手动测试和性能测试评估其在 RISC-V 平台上的可用性。手动测试验证了日常使用场景,测试显示,Chromium 在RevyOS Linux 发行版上具备基本的浏览网页能力,在openEuler上存在稳定性问题,受限于硬件和软件情况,其他系统无法完整运行或缺少支持。性能评估结果显示,RISC-V 平台上的 Chromium 在 Speedometer 3 和 Basemark 基准测试中表现不佳,特别是在需要高性能的视频播放任务上。这表明尽管 Chromium 已初步支持 RISC-V 平台,但在功能完整性和性能优化上仍有改进空间。
\end{abstract}

\thispagestyle{empty} % 首页不显示页码

%%--------------------------目录页------------------------%%
\newpage
\tableofcontents

%%------------------------正文页从这里开始-------------------%
\newpage

\begin{markdown}

# 简介

## 软件说明
Chromium 是一个免费的开源网络浏览器项目,主要由 Google 开发和维护。它是一个广泛使用的代码库,为 Google Chrome 和许多其他浏览器(包括 Microsoft Edge、Samsung Internet 和 Opera)提供了绝大多数代码。该代码还被多个应用程序框架使用。

## 测试目的
本次测试皆在验证 Chromium 在 RISC-V 平台上的可用性,特别是在 Milk-V Pioneer Box 和 Sipeed LicheePi 4A 两个典型平台上的表现。通过手动测试的方法,评估了 Chromium 当前在 RISC-V 平台上的可用性,对目前的平台兼容性,及用户的日常使用体验,给出了定性和定量的结论,并为其未来进一步的优化和支持提供参考。

## 测试概述
本次测试在 RISC-V 设备 Milk-V Pioneer Box 和 Sipeed LicheePi 4A 的多个 Linux 发行版上对多个版本的 Chromium 进行了手动测试,手动测试采用贴近真实用户日常使用场景的测试用例,模拟一般用户的操作方式对 Chromium 在测试环境上的日常浏览体验进行了评估,对其目前在 RISC-V 上的可用性进行了较为全面的测试并得出了相应的结论。

在部分系统下还使用了 Speedometer3 和 Basemark 来测试浏览器浏览网页以及图形性能。

## 测试总结
本次测试在以下平台和系统上验证情况如下:

| 平台        | 发行版        | 测试结果              |
|-------------|---------------|-----------------------|
| Pioneer Box | RevyOS        | 可用                  |
| LicheePi 4A | RevyOS        | 可用                  |
| Pioneer Box | openEuler     | 可用(存在稳定性问题)   |
| LicheePi 4A | openEuler     | 无 GUI 版本           |
| LicheePi 4A | Fedora        | 当前无网卡支持        |
| Pioneer Box | Fedora        | 软件源内没有 Chromium |
| Both        | openKylin     | 软件源内没有 Chromium |
| Both        | openCloudOS   | 未找到可用版本        |
| Both        | Ubuntu        | 未找到可用版本        |
| Both        | Debian | 使用 RevyOS 替代      |


性能测试结果如下:

- Speedometer 3:

| Pioneer Box | LicheePi 4A | x86-64 参考 |
|-------------|-------------|-------------|
|    1.03     | 0.706       | 10-20       |

- Basemark:

| Pioneer Box             | LicheePi 4A             | x86-64 参考 |
|-------------------------|-------------------------|-------------|
| 22.87(部分测试未运行) | 16.98(部分测试未运行) | 2000        |

# 环境说明

## 硬件环境
本次测试主要在 Milk-V Pioneer Box 和 Sipeed LicheePi 4A 上进行,机器硬件配置为:

Milk-V Pioneer Box:
- CPU: SG2042 64 Core C920@2.0GHz
- RAM: 4 channel 3200Hz 128GB DDR4 SODIMM (32GB * 4)
- SSD: PCIe 3.0 x 4 1TB
- GPU: AMD R5 230

Sipeed LicheePi 4A:
- CPU: TH1520, RISC-V 2.0G C910 x4
- RAM: 16 GB 64bit LPDDR4X-3733
- Storage: 128 GB eMMC

## 软件环境

本次测试涵盖的系统版本和 Chromium 版本如下:
- openEuler(仅测试 Pioneer Box):第三方软件仓库自带 Chromium 
  - Pioneer Box 版本: Chromium 128.0.6613.119
- RevyOS:软件仓库自带 Chromium
  - Pioneer Box 版本: Chromium 128.0.6613.113
  - LicheePi 4A 版本: Chromium 109.0.5414.119
  
其余未测试系统及原因如下:

| 平台        | 发行版        | 未测试原因              |
|-------------|---------------|-----------------------|
| LicheePi 4A | openEuler     | 无 GUI 版本           |
| LicheePi 4A | Fedora        | 当前无网卡支持        |
| Pioneer Box | Fedora        | 软件源内没有 Chromium |
| Both        | openKylin     | 软件源内没有 Chromium |
| Both        | openCloudOS   | 未找到可用版本        |
| Both        | Ubuntu        | 未找到可用版本        |
| Both        | Debian | 使用 RevyOS 替代      |

## 环境搭建

### 安装系统

#### Sipeed LicheePi 4A

LicheePi 4A 各个系统在[支持矩阵](https://github.com/ruyisdk/support-matrix/tree/main/LicheePi4A)上详细记载了如何进行安装,可作为参考。

- RevyOS
从 [ISCAS 镜像](https://mirror.iscas.ac.cn/revyos/extra/images/lpi4a/)下载并解压:
```shell!
wget https://mirror.iscas.ac.cn/revyos/extra/images/lpi4a/20240720/boot-lpi4a-20240720_171951.ext4.zst
wget https://mirror.iscas.ac.cn/revyos/extra/images/lpi4a/20240720/u-boot-with-spl-lpi4a.bin
wget https://mirror.iscas.ac.cn/revyos/extra/images/lpi4a/20240720/root-lpi4a-20240720_171951.ext4.zst
zstd -d boot-lpi4a-20240720_171951.ext4.zst
zstd -d root-lpi4a-20240720_171951.ext4.zst
```

按住 boot 键后,上电/Reset 进入刷写模式。

接下来刷写系统:
```shell!
sudo fastboot devices
sudo fastboot flash ram u-boot-with-spl-lpi4a.bin 
sudo fastboot reboot
sudo fastboot flash uboot u-boot-with-spl-lpi4a.bin
sudo fastboot flash boot boot-lpi4a-20240720_171951.ext4
sudo fastboot flash root root-lpi4a-20240720_171951.ext4
```
默认账号密码为:`debian`:`debian`

#### Milk-V Pioneer Box

##### RevyOS
% TODO:烧固件
下载 [RevyOS 20241025](https://mirror.iscas.ac.cn/revyos/extra/images/sg2042/20241025/revyos-pioneer-20241025-001347.img.zst)后,使用 `zstd` 解压, `dd` 到 **NVME 硬盘**中 *(需要一个硬盘盒)*
```shell!
zstd -d revyos-pioneer-20241025-001347.img.zst
sudo dd if=path/to/revyos-pioneer-20241025-001347.img of=/dev/your/nvme bs=4M status=progress
```
将硬盘插入系统后自动启动

##### openEuler
下载[系统镜像](https://mirrors.hust.edu.cn/openeuler/openEuler-24.03-LTS/embedded_img/riscv64/SG2042/openEuler-24.03-LTS-riscv64-sg2042.img.zip),解压,使用 `dd` 烧录至 microSD 卡或者 NVMe SSD。

将下面的 `/dev/sda` 替换成真实硬盘位置。

```shell!
unzip openEuler-24.03-LTS-riscv64-sg2042.img.zip
sudo wipefs -af /dev/sda
sudo dd if=openEuler-24.03-LTS-riscv64-sg2042.img of=/dev/sda bs=1M status=progress
sudo eject /dev/sda
```

### 手动测试环境

#### RevyOS
采用 `apt` 进行安装:
```shell!
sudo apt update
sudo apt upgrade
sudo apt install chromium
```

#### openEuler
openEuler 不自带图形桌面,需要先安装图形环境,以 Gnome 为例:
% TODO: 烧固件
采用 `dnf` 进行安装:

```shell!
sudo dnf update
sudo dnf install dejavu-fonts liberation-fonts gnu-*-fonts

sudo dnf install xorg-x11-apps xorg-x11-drivers xorg-x11-drv-ati \
 xorg-x11-drv-dummy xorg-x11-drv-evdev xorg-x11-drv-fbdev xorg-x11-drv-intel \
 xorg-x11-drv-libinput xorg-x11-drv-nouveau xorg-x11-drv-qxl \
 xorg-x11-drv-synaptics-legacy xorg-x11-drv-v4l xorg-x11-drv-vesa \
 xorg-x11-drv-vmware xorg-x11-drv-wacom xorg-x11-fonts xorg-x11-fonts-others \
 xorg-x11-font-utils xorg-x11-server xorg-x11-server-utils xorg-x11-server-Xephyr \
 xorg-x11-server-Xspice xorg-x11-util-macros xorg-x11-utils xorg-x11-xauth \
 xorg-x11-xbitmaps xorg-x11-xinit xorg-x11-xkb-utils

sudo dnf install adwaita-icon-theme atk atkmm at-spi2-atk at-spi2-core baobab \
 abattis-cantarell-fonts cheese clutter clutter-gst3 clutter-gtk cogl dconf \
 dconf-editor devhelp eog epiphany evince evolution-data-server file-roller folks \
 gcab gcr gdk-pixbuf2 gdm gedit geocode-glib gfbgraph gjs glib2 glibmm24 \
 glib-networking gmime30 gnome-autoar gnome-backgrounds gnome-bluetooth \
 gnome-boxes gnome-builder gnome-calculator gnome-calendar gnome-characters \
 gnome-clocks gnome-color-manager gnome-contacts gnome-control-center \
 gnome-desktop3 gnome-disk-utility gnome-font-viewer gnome-getting-started-docs \
 gnome-initial-setup gnome-keyring gnome-logs gnome-menus gnome-music \
 gnome-online-accounts gnome-online-miners gnome-photos gnome-remote-desktop \
 gnome-screenshot gnome-session gnome-settings-daemon gnome-shell \
 gnome-shell-extensions gnome-system-monitor gnome-terminal \
 gnome-tour gnome-user-docs gnome-user-share gnome-video-effects \
 gnome-weather gobject-introspection gom grilo grilo-plugins \
 gsettings-desktop-schemas gsound gspell gssdp gtk3 gtk4 gtk-doc gtkmm30 \
 gtksourceview4 gtk-vnc2 gupnp gupnp-av gupnp-dlna gvfs json-glib libchamplain \
 libdazzle libgdata libgee libgnomekbd libgsf libgtop2 libgweather libgxps libhandy \
 libmediaart libnma libnotify libpeas librsvg2 libsecret libsigc++20 libsoup \
 mm-common mutter nautilus orca pango pangomm libphodav python3-pyatspi \
 python3-gobject rest rygel simple-scan sushi sysprof tepl totem totem-pl-parser \
 tracker3 tracker3-miners vala vte291 yelp yelp-tools \
 yelp-xsl zenity

```

启动 gdm 显示管理器:
```shell!
sudo systemctl enable gdm --now
```

设置系统默认以图形界面登录:
```shell!
sudo systemctl set-default graphical.target
```

添加 openEuler RISC-V SIG Chromium 源:
```shell!
sudo bash -c 'cat > /etc/yum.repos.d/chromium.repo << EOL
[chromium]
name=chromium
baseurl=https://build-repo.tarsier-infra.isrc.ac.cn/Factory:/RISC-V:/Chromium/24.03/
enabled=1
gpgcheck=0
EOL'
```

安装 chromium:
```shell!
sudo dnf install chromium
```

### 性能测试

- Speedometer 3

在 Chromium 浏览器中访问:https://browserbench.org/Speedometer3.0/

- Basemark

在 Chromium 浏览器中访问:https://web.basemark.com

# 测试内容

## 手动测试

手动测试中,通过模拟用户日常使用习惯,精选了 138 个测试用例进行测试,其涵盖了浏览网页、下载、书签、打印、设置、历史等常见操作,并以此基础进行了实际使用体验和步骤截图。

## 性能测试

- Speedometer 3

Speedometer 3 是 Web 浏览器的基准测试,它通过对各种工作负载上的模拟用户交互进行计时来测量 Web 应用程序响应能力。

- Basemark

Basemark Web 3.0 是一个全面的 Web 浏览器性能基准测试,可测试移动或桌面系统使用基于 Web 的应用程序的能力。该基准测试包括使用网络建议和功能的各种系统和图形测试。

# 测试结果

详细测试数据可见 [Github 仓库](https://github.com/QA-Team-lo/chromium_test)

## 手动测试

手动测试 Chromium 可以正常使用。除解码可能由于没有外设支持导致无法硬解外,其余工作正常。

包括 Google、Baidu、Bilibili 和知乎等网站均能正常访问,账号功能正常可以登录和同步,书签、历史记录、下载等常用功能运行良好。

由于只有软解且 CPU 性能不足,视频播放遇到卡顿,在预期之中。

对于openEuler上的chromium来说,在测试过程中多次崩溃,存在稳定性问题。

## 性能测试

- Speedometer 3

在RevyOS上,Pioneer Box上得分`1.03`分,LicheePi 4A 得分`0.706`分,两台设备均比同系统的Firefox成绩高。作为参考,x86_64电脑得分为10-20分。

- Basemark

LicheePi 4A 和 Pioneer Box 不支持测试所需 WebGL 图形渲染,部分测试项目会产生测试异常。在已完成的测试项目中,Pioneer Box上得分`22.87`分,LicheePi 4A 得分`16.98`分。作为参考,在配备了独立显卡的x86_64电脑上得分为2000分。

# 总结

本次报告评估了 Chromium 浏览器在 RISC-V 平台,包括 Milk-V Pioneer Box 和 Sipeed LicheePi 4A 上的可用性和性能。手动测试中,Chromium在部分分发版本上表现正常,能够满足基本的浏览需求。然而,由于部分硬件和软件支持的限制,某些平台和操作系统无法完整稳定运行 Chromium,和缺少 GUI 版本或者没有软件源支持等。此外,性能测试表明,RISC-V 平台上的 Chromium 在 Speedometer 3 和 Basemark 基准测试中的得分较低。在性能需求较大的场景表现不佳,此外对硬件编解码的支持亦有待完善。
这些结果指出,尽管 Chromium 在 RISC-V 平台上已具备一定可用性,但仍需进一步优化和支持,以提高其功能完整性和性能表现。

\end{markdown}

\newpage
\section{附录}

\appendix

以下是手动测试用例,结果见 \href{https://github.com/QA-Team-lo/chromium_test}{Github 仓库}

\begin{itemize}
    \item 下载 Chromium 中的网页以供离线阅读
    \item 下载二维码分享网页
    \item 从导航面板查找书签
    \item 使用书签栏添加文件夹
    \item 使用全屏模式
    \item 使用右键翻译页面
    \item 使用安全连接查找网站的 IP 地址
    \item 使用已保存的密码登录
    \item 使用标签快捷打开网页
    \item 使用群组整理标签页
    \item 保存网页
    \item 修改书签
    \item 修改标签页分组
    \item 修改网站搜索快捷字词
    \item 停用网站搜索快捷字词
    \item 允许改用干扰性更低的通知方式(禁止网站发送通知,以免干扰)
    \item 允许某个网站发送通知
    \item 关闭 Chromium 拼写检查功能
    \item 关闭书签栏
    \item 关闭“提示保存密码”功能
    \item 关闭是否翻译此语言的网页
    \item 关闭标签页
    \item 关闭自动登录功能
    \item 关闭访客模式
    \item 分享网页
    \item 创建Google账号
    \item 删除书签
    \item 删除保存的密码
    \item 删除保存的网页
    \item 删除来自某个时间段内的 Cookie
    \item 删除特定 Cookie
    \item 删除特定Cookie
    \item 删除网站搜索快捷字词
    \item 利用复制粘贴将书签移至文件夹
    \item 利用拖动将书签移至文件夹
    \item 利用键盘快捷键进行搜索
    \item 勾选删除数据
    \item 取消修改书签
    \item 取消固定标签页
    \item 取消导入书签
    \item 取消导出书签
    \item 取消添加新网页
    \item 取消隐藏快捷方式
    \item 右键添加标签页到阅读清单
    \item 固定标签页
    \item 在 Chromium 中分享或链接到引文和文本
    \item 在 Chromium 中搜索打开的标签页
    \item 在 Chromium 中更改视频字幕样式
    \item 在 Chromium 中登录和同步
    \item 在 Chromium 中翻译网页
    \item 在新标签页中打开文件
    \item 在新标签页中打开链接
    \item 在浏览器下载 PDF 文件
    \item 在浏览器下载图片
    \item 在浏览器下载文件
    \item 在浏览器下载视频
    \item 在网页中搜索
    \item 基础拼写检查选项
    \item 增强的拼写检查功能选项
    \item 复制二维码链接分享网页
    \item 复制粘贴书签将书签栏的书签进行排序
    \item 安装Chromium
    \item 导入书签
    \item 导出书签
    \item 导出保存的密码
    \item 将 Chromium 设置为默认浏览器
    \item 将 Chromium 设置重置为默认设置
    \item 将某个标签页移至另一个窗口
    \item 屏蔽通知
    \item 开启 Chromium 拼写检查功能
    \item 开启 HTTPS-First 模式
    \item 开启书签
    \item 开启关闭退出 Chromium 时清除 Cookie 及网站数据
    \item 开启同步
    \item 开启“提示保存密码”功能
    \item 开启无痕模式
    \item 开启是否翻译此语言的网页
    \item 开启自动登录功能
    \item 开启访客模式
    \item 快捷键开启无痕模式
    \item 恢复标签页或窗口
    \item 手动添加新密码
    \item 打开一组特定网页
    \item 打开新的标签页
    \item 拖动书签将书签栏的书签进行排序
    \item 按字母顺序排列书签
    \item 按字母顺序排列书签失败(列表数据只有一条)
    \item 搜索网络信息
    \item 收起收起和展开标签页群组
    \item 放大所有内容
    \item 日后阅读某个网页
    \item 显示保存的密码
    \item 显示、修改、删除或导出已保存的密码
    \item 暂停或取消下载
    \item 更换浏览器颜色
    \item 更改下载内容的默认保存位置
    \item 更改标签页的排列顺序
    \item 更改默认的 PDF 下载权限
    \item 更改默认的下载权限
    \item 更改默认的位置信息设置
    \item 更改默认的广告权限
    \item 更改默认的弹出式窗口和重定向设置
    \item 更新网页
    \item 查找阅读清单
    \item 查看下载完成的文件
    \item 查看保存的网页
    \item 查看历史记录
    \item 查看和管理书签
    \item 查看已下载的文件的列表
    \item 检查已保存的密码
    \item 检查浏览器项目的安全性
    \item 检查网站连接是否安全
    \item 每次打开浏览器新窗口显示为“新标签页”
    \item 每次打开浏览器窗口打开上次浏览的网页
    \item 添加书签
    \item 添加新文件夹
    \item 添加新网页
    \item 添加网站搜索快捷字词
    \item 添加语言
    \item 清除历史记录
    \item 清除所有 Cookie
    \item 清除浏览数据
    \item 禁止某个网站发送通知
    \item 离线阅读网页
    \item 移除某项下载的内容
    \item 管理特定网站的缩放级别
    \item 缩小所有内容
    \item 自定义下载内容的保存位置
    \item 自行选择要同步哪些数据
    \item 自行选择要同步的数据
    \item 视频播放
    \item 设置所有网页的页面内容大小或字号
    \item 设置默认搜索引擎
    \item 跨多个 Chromium 标签页播放音乐或声音
    \item 跨多个 Chromium 标签页观看视频
    \item 选择主页
    \item 选择要同步的信息
    \item 键盘快捷键打开新的标签页
\end{itemize}

\reference

\end{document}
